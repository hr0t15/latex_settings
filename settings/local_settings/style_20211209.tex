%%% ドキュメントクラスの定義
% 数式の表示位置のオプション(デフォルトは「右寄せ」)
%  fleqn : 左寄せ
%  split : 中央寄せ
% 用紙サイズのオプション(デフォルトは「a4paper」)
%  数学書の一般サイズはA5サイズ(148 mm x 210 mm)
%\documentclass[fleqn, a5paper]{ltjsreport}
%\documentclass[fleqn]{ltjsreport}
\documentclass[fleqn, b5paper]{ltjsreport}

%%% フォント設定
\usepackage{luatexja}
\usepackage{fontspec} %luatexja-fontspecを読み込む場合入れなくても読み込まれます。
\usepackage[match]{luatexja-fontspec}

%fontspec
%欧文用の本文
%\setmainfont{}
%サンセリフ
%\setsansfont{}
%等幅
%\setmonofont{}
%luatexja-fontspec
%和文用の本文
%\setmainjfont{}
%ゴシック
%\setsansjfont{}
%\setmonojfontはluatexja-fontspecにmatchオプションを設定すると使えるようになります。
%matchを設定すると、pLATEX2εの動作に近づき、\textrmなどが和文フォントの設定も変えるようになります。
%\setmonojfont{}

%%% 数式
\usepackage[no-math]{luatexja-preset}

\usepackage{footmisc}

% テンプレートを作るときとかに使うダミーテキストを出力
\usepackage{lipsum}

\usepackage{fancyvrb}

%% 部分的に2段組にしたいときの記述
\usepackage{multicol}

%%% パッケージの定義
%% フォント、シンボル
% AMS
\usepackage{amssymb}
\usepackage{amsmath}
\usepackage{mathrsfs} 

\usepackage{bm, bbm}
%\newcommand{\bm}{\mathbf}
%\newcommand{\bmm}{\boldsymbol}

% 複数行コメント
\usepackage{comment}

%% 極めて自由な書式をもつユーザ命令を定義可能にする
\usepackage{xparse}

%% グラフィック汎用パッケージ
\usepackage{graphicx}
\usepackage{xcolor}

%% ハイパーリンク
% LaTeXでタイプセットしたPDFファイルにHTMLと同様なハイパーリンク機能を加える
\usepackage[plainpages,unicode]{hyperref}
% hyperrefパッケージのオプションリスト
\hypersetup{
  % しおりを作るかの設定(デフォルトは true (作成する))
  %bookmarks = true,
  % しおりに節番号などを付与するかの設定(デフォルトはfalse (付与しない))
  bookmarksnumbered = false,
  % しおりのツリーを開くかの設定(デフォルトは false (閉じている))
  bookmarksopen = true,
  % アンカーテキスト(リンクテキスト)の色指定 (デフォルトは black )
  anchorcolor = black,
  % 参考文献リンクの色指定 (デフォルトは green )
  citecolor = green,
  % ローカルファイルリンクの色指定(デフォルトは magenta )
  filecolor = magenta,
  % リンクに色をつけるかの設定(デフォルトは false (色をつけない))
  colorlinks = true,
  % 作成しているpdfファイル内の内部参照リンクの色(デフォルトはred)
  linkcolor = blue,
  % リンクを囲むボックスの色(デフォルトは1 0 0)
  linkbordercolor = 0 0 0,
  % 外部参照しているURLの色(デフォルトは magenta )
  urlcolor = cyan,
}

% URLリンクの反映
\usepackage{url}

%% ソースコード埋め込み
% ソースコードの直接的な埋め込みに対応
\usepackage{listings}

% listings/jlistingパッケージのオプションリスト
\lstdefinestyle{default}{
  %columns=fixed,
  %basewidth=0.5em,
  %% Languages and styles
  % プログラム言語(複数の言語に対応,C,C++も可)
  %language={言語名前},
  %% Figure out the appearance
  % 標準フォント
  %basicstyle = \ttfamily\scriptsize,
  basicstyle = \small\ttfamily,
  % コメント箇所のフォント
  commentstyle = {\small\ttfamily \color[cmyk]{1,0.4,1,0}},
  % 表示する文字のスタイル
  % キーワード、コメント、および文字列以外のスタイルを決定する。
  % 最後のトークンは、\textbf や \underbar のような1つのパラメータコマンドです。
  stringstyle = {\small\ttfamily \color[rgb]{0,0,1}},
  % キーワード(int, ifなど)のフォント
  keywordstyle = {\small\ttfamily \color[cmyk]{0,1,0,0}},
  %% Getting all characters right
  % タブ数の指定
  tabsize = 4,
  % タブ文字の表示
  % 表示させる場合は true , 表示させない場合は false
  showtabs = false,
  % ブランク文字の表示
  % 表示させない場合は 「showspaces = false, showstringspaces = false 」
  % ブランクにて表示させる場合は 「showspaces = true, showstringspaces = false 」
  % ブランクを示す記号にて表示させる場合は 「showspaces = true」, showstringspaces = false 」
  showspaces = false,
  showstringspaces = true,
  %% Line numbers
  % 行番号の位置
  % 変数は none, left, right の3つ
  %numbers = left,
  numbers = none,
  % 行番号の間隔
  stepnumber = 1,
  % 行番号の書体
  numberstyle = \tiny,
  % 行番号と本文の間隔。
  %numbersep = 3pt,
  %% Margins and line shape
  %左の空白の大きさ
  %xrightmargin = 3pt,
  %右の空白の大きさ
  %xleftmargin = 3pt,
  % 行が長くなったときの改行の設定。
  % trueの場合は改行する。
  breaklines = true,
  % breaklines が true のときの自動改行後のインデント量
  % デフォルトでは20[pt]
  breakindent = 20pt,
  %% Frames
  % フレームの設定
  % 「none, leftline, topline, bottomline, lines, single, shadowbox のいずれか」
  % もしくは
  % 「t(top), r(right), b(bottom), l(left) および
  % T(二重線のtop), R(二重線のright), B(二重線のbottom), L(二重線のleft) の組み合わせ」
  % frame=single,
  % 背景色と透過度
  % backgroundcolor={\color[gray]{.95}},
  % 本文からframeまでの間隔(デフォルトは3pt)
  framesep = 3pt,
  % 固定列モードとフレキシブル列モードの両方の単一文字ボックスの幅を設定する。
  % basewidth {0.6em,0.45em}
  basewidth={0.5em, 0.45em},
}

\lstdefinestyle{syntax}{
  frame=Tb,
  backgroundcolor={\color[gray]{.95}},
}

\lstdefinestyle{term}{
  frame=TRBL,
  backgroundcolor={\color[gray]{.80}},
}

\lstdefinestyle{R}{
  language=R,
  backgroundcolor={\color[gray]{.95}},
}
\lstdefinestyle{termR}{
  language=R,
  frame=TRBL,
  backgroundcolor={\color[gray]{.80}},
}
\lstdefinestyle{C++}{
  language=C++,
}


\lstset{escapechar = , style = {default}}

% 引数
% #1 : ファイル名
% #2 : スタイルの指定
% #3 : ラベル
% #4 : captionの指定
\newcommand{\includeCode}[4][default]{\lstinputlisting[caption = #4, label = #3, escapechar = , style = #2]{#1}}

\usepackage{subfiles}

%% TikZ
% 本体の読み込み
\usepackage{tikz}
\usetikzlibrary{intersections, calc, patterns, math, arrows, shapes}
\usepackage{pgfplots}
\pgfplotsset{compat=1.15}
\usepackage{mathrsfs}
\usetikzlibrary {arrows.meta}
\usetikzlibrary {bending}

%% 外枠定義
% 外枠記述のパッケージ
\usepackage{framed}
\usepackage{fancybox}

% tcolorboxパッケージの使用
\usepackage[many]{tcolorbox}
% 凝った内容の枠を作成する場合には,さらに library を読み込む必要があります。
% raster library と skins library を読み込みたい場合には,以下の通り追加する。
\tcbuselibrary{raster, skins}
% ページまたぎに対応したboxにする。
\tcbuselibrary{breakable}
% 複数変数の取得を可能にする。
\tcbuselibrary{xparse, listings, theorems}

% --------------
% the space reserved between for the ``In'' numbers and the code
\newlength\inwd
\setlength\inwd{0.8cm}

\newtcblisting{ipyin}{
  enlarge left by=\inwd,
  width=\linewidth-\inwd,
  enhanced,
  boxrule=0.pt,
  colback=gray!5!white,
  colbacktitle=black!80!white,
  listing only,
  breakable,
  top=0pt,
  bottom=0pt,
  overlay={
    \node[
      anchor=north east,
      text width=\inwd,
      font=\footnotesize\ttfamily\color{blue!50!black},
      inner ysep=2mm,
      inner xsep=0pt,
      outer sep=0pt
      ] 
      at (frame.north west)
      {\ In :};
  }
  listing options={
    basicstyle=\footnotesize\ttfamily,
    language=python,
    escapechar=¢,
    showstringspaces=false,
    },
}

\newtcblisting{ipyout}{
  enlarge left by=\inwd,
  width=\linewidth-\inwd,
  enhanced,
  boxrule=0.pt,
  colback=gray!5!white,
  colbacktitle=black!80!white,
  listing only,
  breakable,
  top=0pt,
  bottom=0pt,
  overlay={
    \node[
      anchor=north east,
      text width=\inwd,
      font=\footnotesize\ttfamily\color{red!50!black},
      inner ysep=2mm,
      inner xsep=0pt,
      outer sep=0pt
      ] 
      at (frame.north west)
      {Out :};
  }
  listing options={
    basicstyle=\footnotesize\ttfamily,
    language=python,
    escapechar=¢,
    showstringspaces=false,
    },
}
% ------------

%% 表組み
\usepackage{float}
\usepackage{multirow}

%% ディスプレイ数式時の改ページ
% 複数行にわたる数式の途中で改ページを許可するコマンド。
% 0~4で指定。
% 0は改ページしない、4は必ず改ページする。
% 改ページしたくない行は\\*とする。
\allowdisplaybreaks[3]

%% レイアウト
\usepackage[top=20truemm,bottom=15truemm,left=15truemm,right=15truemm]{geometry}

\makeatletter
\newcommand{\subsubsubsection}{\@startsection{paragraph}{4}{\z@}%
  {1.0\Cvs \@plus.5\Cdp \@minus.2\Cdp}%
  {.1\Cvs \@plus.3\Cdp}%
  {\reset@font\sffamily\normalsize}
}
\makeatother
%\setcounter{secnumdepth}{4}

\numberwithin{equation}{section}
\newtheorem{Thm}{定理}[section]
\newtheorem{Lemma}[Thm]{補題}
\newtheorem{Def}[Thm]{定義}
\newtheorem{Prop}[Thm]{命題}
\newtheorem{Fact}[Thm]{事実}
\newtheorem{Cor}[Thm]{系}
\newtheorem{Conj}[Thm]{予想}
\newtheorem{Ex}[Thm]{例}
\newtheorem{Axiom}[Thm]{公理}
\newtheorem{Notation}[Thm]{記法}
\newtheorem{Symbol}[Thm]{記号}
\newtheorem{Prob}[Thm]{問題}
\newtheorem{Word}[Thm]{用語}
\newtheorem{Claim}[Thm]{注意}
\newtheorem{Mon}{問題}[section]

\newtheorem{teiri}[Thm]{定理}
\newtheorem{hodai}[Thm]{補題}
\newtheorem{teigi}[Thm]{定義}
\newtheorem{meidai}[Thm]{命題}
\newtheorem{zizitsu}[Thm]{事実}
\newtheorem{kei}[Thm]{系}
\newtheorem{yosou}[Thm]{予想}
\newtheorem{rei}[Thm]{例}
\newtheorem{kori}[Thm]{公理}
\newtheorem{kihou}[Thm]{記法}
\newtheorem{kigou}[Thm]{記号}
\newtheorem{mondai}[Thm]{問題}
\newtheorem{yougo}[Thm]{用語}

\newcommand{\proof}{\noindent{\bf [\hspace{0.3mm}証明\hspace{0.3mm}]\quad }}
\newcommand{\nproof}[1]{\noindent{\bf [\hspace{0.3mm}{#1}の証明\hspace{0.3mm}]\quad }}
\newcommand{\remark}{\noindent{\bf 注意}\quad }
\newcommand{\ans}{\noindent{\bf [\hspace{0.3mm}解答\hspace{0.3mm}]\quad }}
\newcommand{\answer}{\noindent{\bf [\hspace{0.3mm}解答\hspace{0.3mm}]\quad }}
\newcommand{\nans}[1]{\noindent{\bf [\hspace{0.3mm}{#1}の解答\hspace{0.3mm}]\quad }}
\newcommand{\nanswer}[1]{\noindent{\bf [\hspace{0.3mm}{#1}の解答\hspace{0.3mm}]\quad }}
% 証明終了の四角
\newcommand{\qed}{\hfill $\Box$ \vspace{0.8\baselineskip}}

%% 文書の階層定義
% \renewcommand{\thepart}{Roman{part}}
% \def\presubsubsectionname{(}
% \def\postsubsubsectionname{)}
\setcounter{secnumdepth}{3}

% \renewcommand{\thesubsection}{Kanji{subsection}}[section]

% enumerate 環境での番号の形式の変更
%\renewcommand{\labelenumi}{(\arabic{enumi})}
%\renewcommand{\labelenumii}{(\arabic{enumi})}
%\renewcommand{\labelenumiii}{(\arabic{enumi})}
%\renewcommand{\labelenumiiii}{(\arabic{enumi})}

% 高機能番号付き箇条書き
\usepackage{enumerate}


\newcommand{\ds}{\displaystyle}

%% 新しい箇条書きの定義
% enumlist : 定義や命題の列挙で使う
% i), ii), ...
\newenvironment{enumlist}{
  \begin{enumerate}
  \renewcommand{\labelenumi}{\roman{enumi})}
  \renewcommand{\labelenumii}{\roman{enumii})}
  \renewcommand{\labelenumiii}{\roman{enumiii})}
  \renewcommand{\labelenumiv}{\roman{enumiv})}
}{
  \end{enumerate}
}
% proplist : 同値な命題の列挙で使う
% (I), (II), ...
\newenvironment{proplist}{
  \begin{enumerate}
  \renewcommand{\labelenumi}{(\Roman{enumi})}
  \renewcommand{\labelenumii}{(\Roman{enumii})}
  \renewcommand{\labelenumiii}{(\Roman{enumiii})}
  \renewcommand{\labelenumiv}{(\Roman{enumiv})}
}{
  \end{enumerate}
}
% eqlist : 満たすべき条件の列挙で使う
% (1), (2), ...
\newenvironment{eqlist}{
  \begin{enumerate}
  \renewcommand{\labelenumi}{(\arabic{enumi})}
  \renewcommand{\labelenumii}{(\arabic{enumii})}
  \renewcommand{\labelenumiii}{(\arabic{enumiii})}
  \renewcommand{\labelenumiv}{(\arabic{enumiv})}
}{
  \end{enumerate}
}
% steplist : 満たすべき条件の列挙で使う
% step1:, step2:, ...
\newenvironment{steplist}{
  \begin{enumerate}
  \renewcommand{\labelenumi}{step\arabic{enumi}:\ \ }
  \renewcommand{\labelenumii}{step\arabic{enumii}:\ \ }
  \renewcommand{\labelenumiii}{step\arabic{enumiii}:\ \ }
  \renewcommand{\labelenumiv}{step\arabic{enumiv}:\ \ }
}{
  \end{enumerate}
}

%% よく使う集合
\newcommand{\Int}{\operatorname{Int}}
\newcommand{\map}{\operatorname{Map}}
\newcommand{\Image}{\operatorname{Image}}
\newcommand{\ord}{\operatorname{Ord}}
\newcommand{\tr}{\operatorname{tr}}
\newcommand{\diag}{\operatorname{diag}}
\newcommand{\rank}{\operatorname{rank}}

%% ギリシア数字
\newcommand{\gi}{I}
\newcommand{\gii}{I\hspace{-.1em}I}
\newcommand{\giii}{I\hspace{-.1em}I\hspace{-.1em}I}
\newcommand{\giv}{I\hspace{-.1em}V}
\newcommand{\gv}{V}
\newcommand{\gvi}{\hspace{-.1em}I}
\newcommand{\gvii}{V\hspace{-.1em}I\hspace{-.1em}I}
\newcommand{\gviii}{V\hspace{-.1em}I\hspace{-.1em}I\hspace{-.1em}I}
\newcommand{\gix}{I\hspace{-.1em}X}
\newcommand{\gx}{X}

%% 代数体
%bbXY(X:代数体の一般的な略字 Y:全体→なし, 正値全体→p(Positive), 負値全体→n(Negative))
\newcommand{\bbn}{\mathbb{N}}
\newcommand{\bbz}{\mathbb{Z}}
\newcommand{\bbq}{\mathbb{Q}}
\newcommand{\bbr}{\mathbb{R}}
\newcommand{\bbi}{\mathbb{R}$/$\mathbb{Q}}
\newcommand{\bbc}{\mathbb{C}}
\newcommand{\bbk}{\mathbb{K}}
\newcommand{\bbnp}{\mathbb{N}_{+}}
\newcommand{\bbzp}{\mathbb{Z}_{+}}
\newcommand{\bbqp}{\mathbb{Q}_{+}}
\newcommand{\bbrp}{\mathbb{R}_{+}}
\newcommand{\bbip}{\mathbb{R}_{+}$/$\mathbb{Q}_{+}}
\newcommand{\bbnn}{\mathbb{N}_{-}}
\newcommand{\bbzn}{\mathbb{Z}_{-}}
\newcommand{\bbqn}{\mathbb{Q}_{-}}
\newcommand{\bbrn}{\mathbb{R}_{-}}
\newcommand{\bbin}{\mathbb{R}_{-}$/$\mathbb{Q}_{-}}

%% よく使う文字列
\newcommand{\AtoZ}{ABCDEFGHIJKLMNOPQRSTUVWXYZ}
\newcommand{\atoz}{abcdefghijklmnopqrstuvwxyz}
\newcommand{\suuji}{0123456789}

% 微分
% (全)微分
\newcommand{\diff}[2]{\frac{d {#1}}{d {#2}}}
\newcommand{\difff}[2]{\frac{d^{2} {#1}}{d {#2}^{2}}}
% 偏微分
\newcommand{\pdiff}[2]{\frac{\partial {#1}}{\partial {#2}}}
\newcommand{\pdifff}[2]{\frac{\partial^{2} {#1}}{\partial {#2}^{2}}}
\newcommand{\pdiffff}[3]{\frac{\partial^{2} {#1}}{\partial {#2} \partial {#3}}}

% 組み合わせ(Combination)
\newcommand{\comb}[2]{{}_{#1} \mathrm{C}_{#2}}
% 重複組み合わせ
\newcommand{\Comb}[2]{{}_{#1} \mathrm{H}_{#2}}
% 順列 (Permutation)
\newcommand{\perm}[2]{{}_{#1} \mathrm{P}_{#2}}
% 重複順列
\newcommand{\Perm}[2]{{}_{#1} \Pi_{#2}}
% esssup/essinf
\DeclareMathOperator*{\esssup}{ess\,sup}
\DeclareMathOperator*{\essinf}{ess\,inf}
% argmin/argmax
\DeclareMathOperator*{\argmin}{argmin}
\DeclareMathOperator*{\argmax}{argmax}

% 内積の定義
\newcommand{\inner}[1]{\langle {#1} \rangle}
\newcommand{\Inner}[1]{\left\langle {#1} \right\rangle}


%% ローカルの設定の読み込み

\newcommand{\sign}{\operatorname{sign}}
\newcommand{\diag}{\operatorname{diag}}
\newcommand{\supp}{\operatorname{supp}}

\newcommand{\Real}{\mathscr{R}e\ }
\newcommand{\Im}{\mathscr{I}m\ }

% 期待値
\newcommand{\EE}{\mathbb{E}}
% 分散
\newcommand{\Var}{\operatorname{Var}}
% 共分散
\newcommand{\Cov}{\operatorname{Cov}}
% 相関係数
\newcommand{\Corr}{\operatorname{Corr}}

\newcommand{\mode}{\operatorname{mode}}
\newcommand{\vvec}{\operatorname{vec}}

\newcommand{\iid}{\stackrel{i.i.d.}{\scalebox{1.5}[1]{\sim}}}

\newcommand{\bias}{\operatorname{bias}}
\newcommand{\hvs}[2]{H_{0} : {#1}\ \operatorname{vs.}\ H_{1} : {#2}}
\newcommand{\reject}{\Rightarrow \operatorname{reject}}
\newcommand{\pvalue}{p\operatorname{-value}}


%% 離散一様分布(Uniform_d : Uni(N))
\newcommand{\Uni}{\operatorname{Uni}}
%%% ベルヌーイ分布(Bernoulli : Ber(p))
\newcommand{\Ber}{\operatorname{Ber}}
%%% 二項分布(Binomial : Bin(n, p))
\newcommand{\Bin}{\operatorname{Bin}}
%%% ポアソン分布(Poisson : Po(\lambda))
\newcommand{\Po}{\operatorname{Po}}
%%% 幾何分布(Geometric : Geo(p))
\newcommand{\Geo}{\operatorname{Geo}}
%%% 負の二項分布(Neg_Binomial : NBin(r, p))
\newcommand{\NBin}{\operatorname{NBin}}
%%% 超幾何分布(HyperGeometric : HGeo(N, M, n))
\newcommand{\HGeo}{\operatorname{HGeo}}

%% 1次元連続
%%% 連続一様分布(Uniform_c : Uni(a, b))
% 離散と同じ
%%% 正規分布(Normal : \mathcal{N}(\mu, \sigma^{2}))
\newcommand{\NN}{\mathcal{N}}
\newcommand{\Normal}{\mathcal{N}}
%%% ガンマ分布(Gamma : Gamma(\alpha, \beta))
%%% 逆ガンマ分布(Gamma_-1 : Gamma^{-1}(\alpha, \beta))
\newcommand{\Gam}{\operatorname{Gamma}}
\newcommand{\GAMMA}{\operatorname{Gamma}}
%%% 指数分布(Exponential : Exp(\lambda))
\newcommand{\Exp}{\operatorname{Exp}}
%%% ワイブル分布(Weibull : Weibull(\alpha, \beta))
\newcommand{\Weibull}{\operatorname{Weibull}}
%%% ベータ分布(Beta : Beta(a, b))
\newcommand{\BETA}{\operatorname{Beta}}
%%% パレート分布(Pareto : Pareto(\alpha, \beta))
\newcommand{\Pareto}{\operatorname{Pareto}}
%%% 対数正規分布(LogNormal)
% \newcommand{\}{\operatorname{}}
%%% ロジスティック分布(Logistic : Logistic())
\newcommand{\Logistic}{\operatorname{Logistic}}

%% 多次元離散
%%% 多項分布(Multinomial : Multi(n, p))
\newcommand{\Multi}{\operatorname{Multi}}
%%% ディリクレ分布(PRML)(Dirichlet : Dir())
\newcommand{\Dir}{\operatorname{Dir}}
%%% 多次元超幾何分布(余裕があれば)
% \newcommand{\}{\operatorname{}}

%% 多次元連続
%%% 多次元正規分布(d-dim_Normal : \mathcal{N}(\mu, \Sigma))
% 1次元と同じ
%%% ウイシャート分布(Wishart : Wis())
%%% 逆ウイシャート分布(Wishart_-1 : Wis^{-1}())
\newcommand{\Wis}{\operatorname{Wis}}

\newcommand{\KL}{\operatorname{KL}}
\newcommand{\AIC}{\operatorname{AIC}}
\newcommand{\BIC}{\operatorname{BIC}}

